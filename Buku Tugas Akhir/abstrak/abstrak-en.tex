\begin{center}
  \large\textbf{ABSTRACT}
\end{center}

\addcontentsline{toc}{chapter}{ABSTRACT}

\vspace{2ex}

\begingroup
  % Menghilangkan padding
  \setlength{\tabcolsep}{0pt}

  \noindent
  \begin{tabularx}{\textwidth}{l >{\centering}m{3em} X}
    % Ubah kalimat berikut dengan nama mahasiswa
    \emph{Name}     &:& Charles Chang \\

    % Ubah kalimat berikut dengan judul tugas akhir dalam Bahasa Inggris
    \emph{Title}    &:& \emph{Multi-Modal Person Re-Identification using Lightweight Convolutional Neural Network} \\

    % Ubah kalimat-kalimat berikut dengan nama-nama dosen pembimbing
    \emph{Advisors} &:& 1. Dr. Reza Fuad Rachmadi, S.T., M.T. \\
                    & & 2. Dr. I Ketut Eddy Purnama, S.T, M.T \\
  \end{tabularx}
\endgroup

% Ubah paragraf berikut dengan abstrak dari tugas akhir dalam Bahasa Inggris
\emph{As a complement to security systems, CCTVs are increasingly used to monitor and analyze criminal acts done at a given location. However, the manual search for criminals is still prone to human error. One of the solutions to make the process more effective and efficient is with the use of re-identification.}

\emph{Re-identification is a computer vision and deep learning technique in which an anonymized identity of an image is matched with its owner. In this paper, we will study the method of re-identifying people with multi-modal images where the query is in the form of a body sketch drawn by several different artists.}

\emph{Re-identification techniques in this book are implemented using lightweight Convolutional Neural Network, namely the Residual Network used to classify the CIFAR10 dataset. In this research, we achieved a Rank-1 accuracy of 21\% with our ensemble model, and 19.8 \% with our ResNet 110 model.}

% Ubah kata-kata berikut dengan kata kunci dari tugas akhir dalam Bahasa Inggris
\emph{Keywords}: \emph{Re-Identification}, \emph{Multi-Modal}, \emph{Criminal}.
