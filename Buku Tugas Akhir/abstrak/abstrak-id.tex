\begin{center}
	\Large\textbf{ABSTRAK}
\end{center}
\vspace{1ex}

\begin{adjustwidth}{-0.2cm}{}
	\begin{tabular}{lcp{0.6\linewidth}}
		Nama Mahasiswa &:& Charles Chang \\
		Judul Tugas Akhir &:& Re-identifikasi Orang menggunakan
		Lightweight Convolutional Neural Network
		pada Multi-Modal Image\\
		Pembimbing &:& 1. Dr. Reza Fuad Rachmadi, S.T., M.T. \\
		& & 2. Dr. I Ketut Eddy Purnama, S.T, M.T.  \\
	\end{tabular}
\end{adjustwidth}
\vspace{1ex}

\setlength{\parindent}{0cm} Sebagai alat pelengkap keamanan, sistem CCTV semakin banyak digunakan di setiap ruang publik untuk memantau dan menganalisa tindakan kriminal pada suatu lokasi. Akan tetapi, pencarian kriminal secara manual masih rentan akan kesalahan manusia. Salah satu solusi membuat pencarian kriminal lebih efektif dan efisien adalah dengan penggunaan Re-identifikasi.
\par  Re-identifikasi merupakan sebuah teknik visi komputer dan deep learning dimana dilakukan pencarian ulang citra atau video milik sebuah identitas.Pada tugas akhir ini, akan dipelajari metode Re-identifikasi orang dengan citra multi-modal, adapun data Input yang akan digunakan berupa sketsa tubuh yang digambar oleh beberapa seniman berbeda.
\par  Teknik yang dipelajari akan diimplementasikan menggunakan Lightweight Convolutional Neural Network dan pre-trained weights dari model yang digunakan untuk mengklasifikasi dataset CIFAR-10. Hasil yang diharapkan melalui Tugas Akhir ini adalah terciptanya sebuah model yang dapat melakukan Re-identifikasi orang riil dari input sketsa menggunakan Lightweight Convolutional Neural Network, sehingga pencarian kriminal di Indonesia dapat dilakukan dengan lebih efisien. Pada penelitian ini kami mendapatkan akurasi Rank-1 sebesar 21\% dengan menggunakan model ensemble kami, serta mendapatkan akurasi Rank-1 sebesar 19.8\% dengan menggunakan model ResNet110.

\vspace{2ex}

Kata Kunci : Re-identifikasi, Multi-Modal, Kriminal
\newpage