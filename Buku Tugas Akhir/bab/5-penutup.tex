\chapter{KESIMPULAN DAN SARAN}
\label{chap:penutup}

% Ubah bagian-bagian berikut dengan isi dari penutup

\section{Kesimpulan}
\label{sec:kesimpulan}

Berdasarkan hasil pengujian yang telah dilakukan dapat ditarik beberapa kesimpulan sebagai berikut:

\begin{enumerate}[nolistsep]

  \item Dikarenakan sketsa \textit{full-body} yang digunakan tidak memiliki warna, dilakukan dua percobaan untuk membantu model melakukan re-identifikasi,yaitu dengan CycleGAN dan Local Binary Pattern. Dengan menghasilkan citra sintesis dari input citra sketsa, CycleGAN meningkatkan Rank-1 \textit{accuracy} dan mAP dari model. Namun percobaan dengan {Local Binary Pattern} yang bertujuan untuk mengambil fitur tekstur dari input sketsa menyebabkan performa dari model menurun di semua metrik sehingga tidak direkomendasikan untuk digunakan.
  
  \item Untuk mengatasi kurangnya data pada PKU Sketch Re-ID, \textit{pre-training} pada dataset Market-1501 dapat meningkatkan semua metrik evaluasi dikarenakan model dapat mempelajari fitur-fitur yang terdapat pada dataset Market 1501 yang memiliki jauh lebih banyak data dibandingkan pada PKU Sketch Re-ID.
  
  \item Penggunaan \textit{Fully Connected Layer} sebesar 512 merupakan yang terbaik, apabila menggunakan \textit{Fully Connected Layer} yang lebih besar dari 512 maka akan terjadi overfitting. Namun apabila menggunakan \textit{Fully Connected Layer} dibawah 512, maka tidak se-optimal 512 layer. 
  
  \item Penggunaan \textit{Random Erasing} membuat performa model tidak konsisten, dari studi ablasi yang dilakukan lebih baik apabila tidak menggunakan \textit{Random Erasing}.
  
  \item Model yang digunakan memiliki kecepatan re-identifikasi 5 kali lebih cepat dan metrik evaluasi yang lebih baik dari DenseNet, sebuah model klasikal yang memiliki parameter lebih banyak 4 kali lipat dari ResNet CIFAR10, dan Triplet SN, sebuah model yang dibuat dari tiga model Sketch-a-Nets dan di optimisasikan dengan triplet loss.

  \item Meskipun bukan merupakan model yang terbaik, model mampu mendapatkan informasi dengan jauh lebih efisien dibandingkan model-model lain. Bahkan model ensemble, model yang paling berat pada penelitian ini, memiliki kecepatan 3 kali lebih cepat dibandingkan model DenseNet dan TripletSN.
  

\end{enumerate}

\section{Saran}
\label{chap:saran}

Untuk pengembangan lebih lanjut pada Tugas Akhir ini, terdapat beberapa saran yang dapat dilakukan, yakni sebagai berikut: 

\begin{enumerate}[nolistsep]

  \item Apabila dataset yang digunakan lebih pemasangan sketch dengan dataset membutuhkan kekuatan komputasi yang lebih besar sehingga tidak disarankan untuk menggunakan CycleGAN.
  
  \item Penggunaan Cross Domain Adversarial Learning untuk menjembatani perbedaan modalitas antara citra sketsa dengan citra CCTV.

  \item Apabila pada dataset  terdapat lebih banyak data maka \textit{pre-training} pada dataset Market 1501 tidak perlu dilakukan.

  \item Melakukan studi ablasi lebih lanjut dengan melakukan perubahan pada ukuran citra input, Random Rotation, dan Batch Size.

\end{enumerate}
