\chapter{PENDAHULUAN}
\pagenumbering{arabic}
\vspace{1ex}

\section*{}
Penelitian ini di latar belakangi oleh berbagai kondisi yang menjadi acuan. Selain itu juga terdapat beberapa permasalahan yang akan dijawab sebagai luaran dari penelitian.
\vspace{1ex}

\section{Latar belakang}
\vspace{1ex}

Re-identifikasi manusia merupakan sebuah teknik visi komputer dan deep learning dimana pada sebuah lingkungan yang terdapat beberapa kamera pengawas dilakukan adanya pencocokan citra seseorang, kepada citra yang ditangkap pada kamera lain. Masalah utama yang ditangani oleh Re-identifikasi manusia dapat disimpulkan sebagai cara untuk mencari representasi diskriminatif milik individu yang ingin dicari. Dengan membuat sistem re-identifikasi manusia, pemeriksaan hasil rekaman yang dilakukan oleh pihak kepolisian dapat dilakukan dengan jauh lebih cepat, dan dapat menekan biaya yang digunakan untuk membayar tenaga kerja. Tidak hanya  untuk melakukan pencarian pelaku tindak kriminal, re-identifikasi dapat digunakan juga untuk melihat apakah barang yang tertinggal pada suatu lokasi diambil individu yang sama, dan dapat membantu pihak sekuriti mencari orang yang hilang. Re-identifikasi manusia dapat mempermudah aktivitas - aktivitas yang sebelumnya dilakukan secara manual, maka dari itu pencarian individu sangat dibutuhkan dengan menggunakan teknologi \textit{machine learning} untuk mempercepat proses yang sebelumnya memakan waktu yang sangat lama.

Namun untuk melakukan re-identifikasi manusia dibutuhkan citra dari pelaku, yang tidak selalu siap tersedia dimiliki oleh pihak kepolisian. Maka dari itu diperlukan adanya re-identifikasi yang menggunakan sketsa sebagai input dari model. Meskipun memiliki konsep yang mirip dengan \textit{Face Recognition}, dikarenakan re-identifikasi sketsa menggunakan gambar \textit{full-body}, terdapat tambahan kompleksitas yang harus dihadapi oleh model yang dibuat. Selain itu masalah ini sangat menantang dikarenakan sketsa tidak memiliki warna sehingga informasi yang didapatkan oleh model lebih sedikit dibanding pada re-identifikasi manusia. 

Penelitian ini telah dilakukan sebelumnya oleh Lu Pang et al. dengan menggunakan model yang canggih dan \textit{cross-domain adversarial learning}, dimana model yang dibuat dapat mencapai Rank-1 akurasi sebesar 34\%. Pada penelitian ini dataset yang digunakan merupakan dataset satu-satunya yang tersedia untuk melakukan re-identifikasi sketsa, yaitu PKU Sketch Re-ID, sebuah dataset yang memuat 200 individu, dimana setiap individu memiliki 3 buah citra, yaitu dua buah citra CCTV dan sebuah citra sketsa. Dataset ini merupakan dataset yang sangat kecil apabila dibandingkan dengan dataset-dataset Person Re-Identification lainnya. Sebagai contoh, dataset Market-1501 memiliki total citra reidentifikasi sebanyak 15535 citra, yaitu 25.8 kali lebih banyak dibandingkan data yang dimiliki oleh dataset PKU Sketch Re-ID. Dikarenakan keterbatasan data yang dimiliki oleh dataset PKU Sketch Re-ID, dataset satu-satunya yang menggunakan citra sketsa \textit{full-body} sebagai input dari model, re-identifikasi sketsa ke manusia masih sangat sulit untuk dilakukan. 

Selain itu penelitian lebih lanjut menggunakan model \textit{lightweight} masih belum pernah dilakukan. Penggunaan model lightweight sangatlah berguna pada kasus \textit{Edge Computing} dimana komputasi dilakukan sedekat mungkin dengan pengguna, sehingga mengurangi waktu respon dan menghemat \textit{bandwidth}. Pada umumnya untuk mengolah data yang sangat besar seperti data dari CCTV dengan cepat, dibutuhkan sebuah server pusat yang sangat dekat dengan pengguna untuk mengatasi isu latensi, namun dengan adanya edge computing, pihak kepolisian dari masing-masing daerah dapat melakukan re-identifikasi pada \textit{edge data center} yang tersedia pada daerahnya. Sehingga dapat menghindari resiko \textit{downtime} maupun latensi yang terjadi apabila komputasi dilakukan pada server pusat.

\pagebreak

Namun Edge Computing tidak lepas dari kekurangan, dikarenakan dilakukan sedekat mungkin dengan pengguna, kekuatan perangkat pada Edge Computing sangatlah terbatas. Dibutuhkan sebuah model \textit{lightweight} sehingga model dapat di jalankan pada perangkat yang memiliki kekuatan komputasi yang tidak terlalu besar. 

Oleh karena itu, pada penelitian ini kami memilih untuk menggunakan model \textit{lightweight}. Model \textit{lightweight} yang kami gunakan merupakan model Residual Network CIFAR-10 dimana Fully Connected layer terakhir dihapus, dan ditambahkan dua buah Fully-Connected layer baru dengan konfigurasi berbeda dengan Fully-Connected layer yang sebelumnya. 

Dengan adanya penelitian ini, diharapkan hasil re-identifikasi sketsa dapat diolah dan dikembangkan untuk membantu pencarian pelaku tindak kriminal oleh pihak kepolisian di Indonesia.

\vspace{1ex} 
\section{Permasalahan}
\vspace{1ex}
Berdasarkan data yang telah dipaparkan di latar belakang, dapat dirumuskan beberapa rumusan masalah sebagai berikut:
\begin{enumerate}
	\vspace{-1.3mm}
	\item Data hasil rekaman kamera masih diperiksa secara manual oleh pihak kepolisian, dari data yang diambil dari Kedeputian Pelayanan Publik Indonesia, dibutuhkan waktu 8 jam untuk melakukan pengecekan sebuah rekaman CCTV \cite{cit:3}. Oleh karena itu dibutuhkan metode pengecekan yang lebih cepat.
	\vspace{-2mm}
	\item Dataset PKU Sketch Re-ID yang digunakan memiliki jumlah data yang relatif sedikit. Jumlah total citra yang dimiliki oleh PKU Sketch Re-ID hanyalah 3.86\% dari total citra dataset Market 1501, dataset yang sangat umum digunakan untuk re-identifikasi sketsa.
	\vspace{-2mm}
	\item Dikarenakan terdapat perbedaan modalitas pada citra sketsa dan citra CCTV, dimana intepretasi citra sketsa masing-masing individu berbeda-beda dikarenakan dilukis oleh seniman yang berbeda beda. Selain itu, sketsa yang digunakan, tidak memiliki warna, sehingga fitur yang dapat digunakan oleh model untuk melakukan re-identifikasi hanyalah fitur tekstur dan warna.
	\vspace{-2mm}
	\item Dibutuhkan sebuah metode yang dapat melakukan re-identifikasi dengan cepat dan tidak membutuhkan kekuatan komputasi yang besar sehingga dapat digunakan pada \textit{edge computing} atau pada \textit{end device}.
\end{enumerate}

\section{Tujuan}
\vspace{1ex}

Adapun tujuan dari penelitian Tugas Akhir ini adalah mengembangkan model re-identifikasi orang menggunakan \textit{Lightweight Convolutional Neural Network}, guna menghasilkan model re-identifikasi sketsa yang lebih efisien akan kekuatan komputasi, sehingga dapat digunakan pada Edge Computing.

\vspace{1ex}

\section{Batasan masalah}
\vspace{1ex}
Batasan masalah yang timbul dari permasalahan Tugas Akhir ini adalah:
\vspace{1ex}
\begin{enumerate}
	\vspace{-2mm}
	\item Data \textit{training} dan \textit{testing} menggunakan data yang diambil dari PKU SketchRe-ID Dataset.
	\vspace{-2mm}
	\item Jenis re-identifikasi orang yang akan dilakukan adalah \textit{Closed Set} dan \textit{Short Term}, dimana dataset tetap dan tidak bertambah seiring dengan waktu.
	\vspace{-2mm}
	\item Training akan dilakukan menggunakan \textit{Lightweight Convolutional Neural Network} sehingga dapat mengurangi kekuatan komputasi yang dibutuhkan oleh model.
\end{enumerate}
\vspace{1ex}

\section{Manfaat}
\vspace{1ex}
Manfaat dari Tugas Akhir ini sendiri adalah, untuk mempercepat pencarian pelaku tindak kriminal dimana citra milik pelaku tidak dimiliki, sehingga digunakan input sketsa. Selain itu model yang digunakan pada Tugas Akhir ini merupakan model \textit{lightweight}, yang sangat berguna pada rana \textit{Edge Computing}, dimana kekuatan komputasi terbatas.