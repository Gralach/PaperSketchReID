\begin{center}
	\Large\textbf{KATA PENGANTAR}
\end{center}
\vspace{1ex}

\setlength{\parindent}{0.9cm} Puji dan syukur kehadirat Tuhan Yang Maha Esa atas segala karunia-Nya, penulis  dapat menyelesaikan penelitian ini dengan judul \textbf{Re-identifikasi Orang menggunakan Lightweight Convolutional Neural Network pada Multi-Modal Image}.
\vspace{1ex}

Penelitian ini disusun dalam rangka pemenuhan bidang riset di Departemen Teknik Komputer ITS, Bidang Studi \textit{Telematika}, serta digunakan sebagai persyaratan menyelesaikan pendidikan  S1. Oleh karena itu, penulis mengucapkan terima kasih kepada:
\vspace{1ex}

\begin{enumerate}[nolistsep]
	\item Tuhan Yang Maha Esa
	\item Orang tua saya, atas semangat dan segala dukungan yang telah diberikan
	\item Bapak Dr. Reza Fuad Rachmadi, S.T., M.T.
	\item Bapak Dr. I Ketut Eddy Purnama, S.T, M.T.
	\item Bapak-ibu dosen pengajar Departemen Teknik Komputer, atas pengajaran, bimbingan, serta perhatian yang diberikan kepada penulis selama ini.
	\item Sesama asisten lab B201 yang menemani pengerjaan Tugas Akhir.
	\item Serta teman - teman angkatan 2017 yang telah bersama - sama melalui kehidupan perkuliahan bersama penulis
\end{enumerate}
\vspace{1ex}

Kesempurnaan hanya milik Tuhan Yang Maha Esa, untuk itu penulis memohon segenap kritik dan saran yang  membangun. Semoga penelitian ini dapat memberikan manfaat bagi kita semua. Amin.
\begin{flushright}
	\begin{tabular}[b]{c}
		Surabaya, April 2021
		\\
		\\
		Penulis
	\end{tabular}
\end{flushright}